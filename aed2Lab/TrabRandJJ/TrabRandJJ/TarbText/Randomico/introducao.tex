A geração de números aleatórios tem muitos usos (em sua maioria em estatística, para amostragem aleatória e simulação). Antes da computação moderna, pesquisadores que precisavam de números aleatórios os gerava através de vários meios (dado, cartas, roleta, etc.), ou utilizavam as tabelas de números aleatórios existentes.
A primeira tentativa de prover para os pesquisadores um suprimento pronto de dígitos aleatórios foi feita em 1927, quando a Cambridge University Press publicou uma tabela de 41.600 dígitos desenvolvida por Leonard H.C. Tippet. Em 1947, a RAND Corporation gerou números por meio de uma simulação eletrônica de uma roleta; os resultados foram eventualmente publicados em 1955 como A Million Random Digits with 100,000 Normal Deviates (Um milhão de dígitos aleatórios com 100.000 desvios normais).\cite{Sit}

John von Neumann foi um pioneiro dos geradores de números aleatórios baseados em computadores. Um contribuidor notável no campo da geração de números pseudoa leatórios para uso prático, é um matemático paquistanês Dr. Arif Zaman. Em 1951, Derrick Henry Lehmer inventou o gerador linear congruente, utilizado na maioria dos geradores de números pseudoa leatórios atuais. Com a disseminação do uso dos computadores, geradores de números pseudoa leatórios substituíram as tabelas numéricas, e "verdadeiros" geradores aleatórios (hardwares geradores de números pseudoa leatórios) são utilizados apenas em alguns casos.\cite{Sit}

Uma variável pseudo aleatória é uma variável que é criada por um procedimento determinístico (frequentemente um programa de computador ou uma subrotina) que (geralmente) recebe bits aleatórios como entrada. A cadeia pseudoaleatória irá, tipicamente, ser maior do que a cadeia aleatória original, porém menos aleatória (menor entropia, no sentido aplicado na teoria da informação). Isto pode ser útil para algoritmos aleatórios.
Geradores de números pseudoa leatórios são amplamente utilizados em aplicações como modelagem computacional (e.g., Cadeias de Markov), estatística, design experimental, etc. Alguns deles são suficientemente aleatórios para serem úteis nestas aplicações; muitos não são, e uma sofisticação considerável é necessária para determinar corretamente a diferença para qualquer propósito em particular. O uso não-precavido de geradores de números pseudoa leatórios prontamente disponíveis tem causado danos consideráveis, e por muito tempo sustentados, no valor de um grande número de projetos de pesquisas por muitos anos.\cite{Sit}